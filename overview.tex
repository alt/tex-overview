\documentclass{scrartcl}

\usepackage{
graphicx,
cooltooltips,
dtklogos,
tikz,
libertine
%xltxtra
}

\def\XeT{XET}
\def\Aleph{Aleph}
\def\XeLaTeX{XeLa\TeX}
%\setmainfont{Linux Libertine}

\def\mynode#1#2#3#4{
  \node (#1) at (#2) {
    \cooltooltip{#1}{}{#3}{}{#4\strut}
  };
}

\begin{document}

\begin{tikzpicture}
\mynode{tex}{7,2}{born in 1978}{\TeX}

\mynode{xet-tex}{3,1}{the first extension to \TeX, 1987. It was able to typeset in two directions, but only with a mark in the dvi to change the direction.}{\TeX-\XeT}
\draw (tex) to (xet-tex);

\mynode{xet--tex}{3,0}{\TeX-\relax-\XeT\ was able to really put the glyphs on the right place in the dvi.}{\TeX-\relax{}-\XeT}
\draw (xet-tex) to (xet--tex);

\mynode{tex3}{7,0}{Now handling 8-bit input. 1989}{\TeX3};
\draw (tex) to (tex3);

\mynode{omega}{2,-2}{Support for unicode-input. Still constrained on the output}{$\Omega$};
\draw (tex3) to (omega);

\mynode{etex}{4,-2}{„The“}{$\varepsilon$-\TeX};
\draw (xet--tex) to (etex);

\mynode{pdftex}{6,-2}{}{pdf\TeX};
\draw (tex3) to (pdftex);

\mynode{texgx}{8,-2}{}{\TeX{}GX};
\draw (tex3) to (texgx);

\mynode{nts}{10,-2}{}{\NTS};
\draw (tex3) to (nts);

\mynode{aleph}{3,-4}{}{\Aleph};
\draw (omega) to (aleph);
\draw (etex) to (aleph);

\mynode{xetex}{8,-4}{}{\XeTeX};
\draw (texgx) to (xetex);
\draw (etex) to (xetex);

\mynode{extex}{10,-4}{}{εχ\TeX};
\draw (nts) to (extex);

\mynode{pdfetex}{5,-4}{}{pdf($\epsilon$)-\TeX};
\draw (etex) to (pdfetex);
\draw (pdftex) to (pdfetex);

\mynode{luatex}{4,-6}{}{Lua\TeX};
\draw (aleph) to (luatex);
\draw (pdfetex) to (luatex);

\end{tikzpicture}

\vspace*{3cm}

\begin{tikzpicture}
\node (latex209) at (0,0) {\LaTeX\ 2.09};

\node (latex2ε) at (0,-2) {\LaTeXe};
\draw (latex209) to (latex2ε);

\node (pdflatex) at (2,-4) {pdf\LaTeXe};
\draw (latex2ε) to (pdflatex);

\node (xelatex) at (4,-4) {\XeLaTeX};
\draw (latex2ε) to (xelatex);

\node (lualatex) at (6,-4) {Lua\LaTeX};
\draw (latex2ε) to (lualatex);

\mynode{lambda}{8,-4}{A LaTeX-package for omega.}{$\Lambda$}
\draw (latex209) to (lambda);

\node (latex3) at (0,-6) {\LaTeX{}3};
\draw (latex2ε) to (latex3);
\end{tikzpicture}

\end{document}