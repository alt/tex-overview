\documentclass{scrartcl}

\usepackage{
graphicx,
cooltooltips,
dtklogos,
tikz,
libertine
%xltxtra
}

\def\XeT{XET}
\def\Aleph{Aleph}
\def\XeLaTeX{XeLa\TeX}
%\setmainfont{Linux Libertine}

\def\mynode#1#2#3#4{
  \node (#1) at (#2) {
    \cooltooltip{#1}{#3}{#3}{}{#4\strut}
  };
}
\title{A short overview of \TeX\ and its children\dots}
\author{Arno Trautmann\thanks{arno.trautmann@gmx.de -- send me any suggestions and comments!}}
\date{}
\begin{document}
\maketitle

This paper tries to give a short overview of the development of \TeX. So far, most of the information is from the article \textsf{A brief history of \TeX, volume II} by Arthur Reutenauer in \textsf{EuroBacho\TeX 2007}.

\begin{tikzpicture}
\mynode{tex}{7,2}{born in 1978}{\TeX}

\mynode{xet-tex}{3,1}{The first extension to TeX, 1987. It was able to typeset in two directions, but only with a mark in the dvi to change the direction.}{\TeX-\XeT}
\draw (tex) to (xet-tex);

\mynode{xet--tex}{3,0}{TeX--XeT was able to really put the glyphs on the right place in the dvi.}{\TeX-\relax{}-\XeT}
\draw (xet-tex) to (xet--tex);

\mynode{tex3}{7,0}{Now handling 8-bit input. 1989}{\TeX3};
\draw (tex) to (tex3);

\mynode{omega}{2,-2}{Support for unicode-input. Still constrained on the output}{$\Omega$};
\draw (tex3) to (omega);

\mynode{etex}{4,-2}{„The“}{$\varepsilon$-\TeX};
\draw (xet--tex) to (etex);

\mynode{pdftex}{6,-2}{}{pdf\TeX};
\draw (tex3) to (pdftex);

\mynode{texgx}{8,-2}{}{\TeX{}GX};
\draw (tex3) to (texgx);

\mynode{nts}{10,-2}{}{\NTS};
\draw (tex3) to (nts);

\mynode{aleph}{3,-4}{originally named epsilon-Omega, an attempt to stabilize Omega whil merging epsilon extensions.}{\Aleph};
\draw (omega) to (aleph);
\draw (etex) to (aleph);

\mynode{xetex}{8,-4}{}{\XeTeX};
\draw (texgx) to (xetex);
\draw (etex) to (xetex);

\mynode{extex}{10,-4}{}{$\epsilon\chi$\TeX};
\draw (nts) to (extex);

\mynode{pdfetex}{5,-4}{}{pdf($\epsilon$)-\TeX};
\draw (etex) to (pdfetex);
\draw (pdftex) to (pdfetex);

\mynode{luatex}{4,-6}{}{Lua\TeX};
\draw (aleph) to (luatex);
\draw (pdfetex) to (luatex);

\end{tikzpicture}
\newpage
\vspace{3cm}

\begin{tikzpicture}
\mynode{latex209}{0,0}{}{\LaTeX\ 2.09};

\mynode{latex2ε}{0,-2}{}{\LaTeXe};
\draw (latex209) to (latex2ε);

\mynode{pdflatex}{2,-4}{}{pdf\LaTeXe};
\draw (latex2ε) to (pdflatex);

\mynode{xelatex}{4,-4}{}{\XeLaTeX};
\draw (latex2ε) to (xelatex);

\mynode{lualatex}{6,-4}{LaTeX auf LuaTeX basierend}{Lua\LaTeX};
\draw (latex2ε) to (lualatex);

\mynode{lambda}{8,-4}{A LaTeX-package for omega.}{$\Lambda$}
\draw (latex209) to (lambda);

\mynode{latex3}{0,-6}{}{\LaTeX{}3};
\draw (latex2ε) to (latex3);
\end{tikzpicture}
\newpage

\begin{tikzpicture}
\mynode{mkii}{0,0}{}{Con\TeX{}t Mk ii};

\mynode{mkiv}{0,0}{}{Con\TeX{}t Mk iv};
\draw (mkii) to (mkiv);
\end{tikzpicture}


\end{document}